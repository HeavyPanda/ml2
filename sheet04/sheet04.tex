\documentclass[a4paper]{article}
\usepackage{a4wide}
\usepackage{amsmath}
\usepackage{amssymb}
\usepackage{dsfont}
\pagestyle{plain}
\usepackage{tikz}
\usetikzlibrary{matrix}
\usepackage{eurosym}
\usepackage{amsthm}
\usepackage{mathrsfs}
\usepackage{dlfltxbcodetips}
\usepackage{cancel}

%    \usepackage{tkz-graph}  
%    \usetikzlibrary{shapes.geometric}%   
% 	 \usetikzlibrary{bayesnet}
	
%\documentclass[12pt]{article}

\usepackage[utf8]{inputenc} % UTF8-Kodierung für Umlaute usw
% \usepackage[ngerman]{babel} % Sprache: Deutsch
\usepackage{graphicx}

\usepackage[margin=1in]{geometry}
\usepackage{array}
\usepackage{amsthm,amssymb}
%\usepackage[fleqn]{amsmath}

\usepackage{multicol}

\newcommand*{\carry}[1][1]{\overset{#1}}
\newcolumntype{B}[1]{r*{#1}{@{\,}r}}

\newcommand{\N}{\mathbb{N}}
\newcommand{\Z}{\mathbb{Z}}

\newcommand{\extitle}[1]{{\vspace{5mm}\Large\b{#1}}\vspace{2mm}\\}
\newcommand{\exsubtitle}[1]{{\large\b{#1}}\vspace{2mm}\\}

\renewcommand*{\i}[1]{{\textit{#1}}}
\renewcommand*{\b}[1]{{\textbf{#1}}}
\renewcommand*{\tt}[1]{{\texttt{#1}}}

\setlength{\parindent}{0cm}
%\setlength{\mathindent}{0cm}
 % import config
\DeclareMathOperator*{\argmax}{arg\,max}
\DeclareMathOperator*{\argmin}{arg\,min}  
\newcommand{\norm}[1]{\left\lVert#1\right\rVert}
\newcommand{\vivid}{\stackrel{\text{vivid}}{=}}
\newcommand{\icol}[1]{% inline column vector
  \left(\begin{smallmatrix}#1\end{smallmatrix}\right)%
}

\newcommand\indep{\protect\mathpalette{\protect\indeP}{\perp}}
\def\indeP#1#2{\mathrel{\rlap{$#1#2$}\mkern2mu{#1#2}}}

\newcommand{\mlq}{F(\q)}
\newcommand{\si}{s}
\newcommand{\sj}{{s'}}
\newcommand{\p}{p}
\newcommand{\q}{r}
\newcommand{\z}{w}
%\newcommand{\x}{\mathcal{D}}
%\newcommand{\mlq}{-L(\q)}
%\newcommand{\si}{i}
%\newcommand{\sj}{j}
%\newcommand{\p}{p}
%\newcommand{\q}{q}
%\newcommand{\z}{Z}
%\newcommand{\x}{X}

\newcommand{\pxgz}{\p(\x|\z)}
\newcommand{\pzgx}{\p(\z|\x)}
\newcommand{\pxz}{\p(\x,\z)}
\newcommand{\px}{\p(\x)}

\newcommand{\qz}{\q(\z)}
\newcommand{\qi}{\q_\si}
\newcommand{\qj}{\q_\sj}
\newcommand{\qk}{\q_k}
\newcommand{\pz}{\p(\z)}
\newcommand{\pzi}{\p(\z_\si)}
\newcommand{\pzj}{\p(\z_\sj)}
\newcommand{\pzk}{\p(\z_k)}



\newcommand{\zi}{\z_\si}
\newcommand{\zj}{\z_\sj}
\newcommand{\zk}{\z_k}
\newcommand{\dz}{\mathrm{d}\z}
\newcommand{\dzi}{\mathrm{d}\z_\si}
\newcommand{\dzj}{\mathrm{d}\z_\sj}
\newcommand{\dzk}{\mathrm{d}\z_k} 

\newcommand{\E}{\mathbb{E}}





%\newcommand{\propq}{ \underset{ \text{wrt.}\put(1,1){\scriptsize *}q }{\propto} }

\newcommand{\eqs}[1]{ \stackrel{#1}{=}  }
\newcommand{\eqq}{  \overset{\text{!}}{=} }
\newcommand{\propqz}{ \underset{ \overset{\text{wrt.}}{\qz} }{\propto} }
\newcommand{\propz}{ \underset{ \overset{\text{wrt.}}{\z} }{\propto} }
\newcommand{\propzj}{ \underset{ \overset{\text{wrt.}}{\zj} }{\propto} }
%\newcommand{\constw}{ \underset{ \text{ wrt. \w} }{\text{ const}} }
%\newcommand{\consty}{ \underset{ \text{ wrt. \y} }{\text{ const}} }
\newcommand{\const}[1]{{\underset{ \text{ wrt. #1} }{\text{ const}} }}

\newcommand{\tr}[1]{{#1^\top}}
\newcommand{\inv}[1]{{#1^{-1}}}
\newcommand{\trk}[1]{{(#1)^\top}}
\newcommand{\invk}[1]{{(#1)^{-1}}}
\newcommand{\tin}[1]{{(#1^\top)^{-1}}}

\newcommand{\Yt}{{Y^\top}}
\newcommand{\Xt}{{X^\top}}
\newcommand{\xt}{{\x^\top}}
\newcommand{\yt}{{\y^\top}}
\newcommand{\Wt}{{\W^\top}}
\newcommand{\w}{w}
\newcommand{\lx}{{\lambda_{x}}}
\newcommand{\ly}{{\lambda_{y}}}
\newcommand{\rhi}{{r^{(i)}}}
\newcommand{\rhj}{{r^{(j)}}}
\newcommand{\xhi}{{x^{(i)}}}
\newcommand{\shi}{{s^{(i)}}}
\newcommand{\xhj}{{x^{(j)}}}
\newcommand{\shj}{{s^{(j)}}}
\newcommand{\Wi}{{W_{i}}}
\newcommand{\Wj}{{W_{j}}}
\newcommand{\Wit}{W_{i}^\top}
\newcommand{\Wjt}{W_{j}^\top}
\newcommand{\Wij}{W_{ij}}
\newcommand{\Wlk}{W_{lk}}
\newcommand{\ax}{{\alpha_{x}}}
\newcommand{\ay}{{\alpha_{y}}}
\newcommand{\axt}{\alpha_{x}^\top}
\newcommand{\ayt}{\alpha_{y}^\top}
\newcommand{\Cxx}{C_{xx}}
\newcommand{\Cxy}{C_{xy}}
\newcommand{\Cyx}{C_{yx}} 
\newcommand{\Cyy}{C_{yy}}
\newcommand{\1}{\mathds{1}}
\newcommand{\lag}{\mathcal{L}}


\begin{document}

% header configuration
\title{\b{Exercise Sheet 4}}
\author{Machine Learning 2, SS16}

\maketitle

% authors
Mario Tambos, 380599;\quad Viktor Jeney, 348969;\quad Sascha Huk, 321249;\quad Jan Tinapp, 0380549\\


\extitle{Exercise 1 - Sparse Coding}
\textbf{(a)}
\begin{gather*}
	{\partial E\over\partial W}
	=
	{\partial\over\partial W}\eta |W|_F^2
	+
	{\partial\over\partial W} \sum_{i=1}^N (|\xhi - W\shi|^2 +	\lambda |\shi|_1)
	\\
	= 
	\eta\sum_{l}^{d}\sum_{k}^{h}{\partial\over\partial W} (\Wlk)^2
	+ 
	\sum_{i=1}^N
		{\partial\over\partial W}(\xhi - W\shi)^\top (\xhi - W\shi)
	\\
	=
	2\eta W
	+ 
	\sum_{i=1}^N
		-2(\xhi - W\shi)\shi^\top
	=
	2\eta W
	- 
	2\sum_{i=1}^N
		(\xhi - W\shi)\shi^\top			
\end{gather*} 
\textbf{(b)}
\begin{gather*}
	{\partial E\over\partial \shi}
	=
	{\partial\over\partial \shi}\eta |W|_F^2
	+
	{\partial\over\partial \shi} \sum_{j=1}^N (|\xhj - W\shj|^2 +	\lambda |\shj|_1)
	\\
	=
	{\partial\over\partial \shi} (\xhi - W\shi)^\top (\xhi - W\shi) 
	              +	{\partial\over\partial \shi} \lambda |\shi|_1
	\\
	=
	-2 W^\top(\xhi-W\shi) 
	              +	\lambda \sum_{k=1}^h {\partial\over\partial \shi}\shi_k 
	              \quad\quad\text{($\shi_k\geq 0$)}
	\\
	=
	-2 W^\top(\xhi-W\shi) 
	+	\lambda 1_h	
\end{gather*} 
\extitle{Exercise 2 - Sparsifying Non-Linearities}
\textbf{(a)}
The derivative wrt. to W is already equivalent. 
Taking the derivative wrt. $\rhi$ we obtain:
\begin{gather*}
	{\partial\over\partial \rhi}\sum_{j=1}^N 
	(|\xhj - Wg(\rhj)|^2 +	\lambda |\rhj|^2)
	=
	{\partial\over\partial \rhi} |\xhi - Wg(\rhj)|^2 
	+ {\partial\over\partial \rhi} \lambda |\rhi|^2
	\\
	= -2 W^\top(\xhi-Wg(\rhi)){\partial\over\partial \rhi}g(\rhi) 
	+	\lambda 2\rhi	 
\end{gather*}
Comparing the factors of $\lambda$ yields componentwise 
differences by the componentwise factors $2\rhi_k$ $\forall k\in\{1,\ldots,h\}$. In order for 
the two problems to be equal, we choose g such that it's Jacobian is 
diagonal and such that it has the k-th of these factors on the k-th diagonal element.  
Therefore we have an equivalent problem for $g(\rhi) = (\rhi_1^2, \ldots, \rhi_h^2)$.
\\
\\
\textbf{(b)}
Both, the reconstruction error as well as the sparsity penalty, are 
convex in both approaches. Since the sum of convex functions is again a convex 
function we find a global, unique optimum in both problems. 
Unfortunately, the derivative of the original problem wrt. $s_k$ 
for $k\in\{1,\ldots,h\}$ does not exist in 0. It is therefore theoretically 
problematic to use gradient descent on the original problem, even though 
encountering this problem might be rather unlikely. 

\newpage
\extitle{Exercise 3 - Autoencoders}
\textbf{(a)}
Focus on the result of (2a), which is ${\partial E\over\partial \rhi}$. 
Having $\rhi=V^\top\xhi$, by chainrule we get:
\begin{gather*}
{\partial E\over\partial V} 
= {\partial E\over\partial\rhi} {\partial\rhi\over\partial V}
= [-2 W^\top(\xhi-Wg(\rhi)){\partial\over\partial \rhi}g(\rhi) +	
	\lambda 2\rhi] \xhi^\top
\\
= [-2 W^\top(\xhi-Wg(V^\top\xhi))g'(V^\top\xhi) +
	\lambda 2V^\top\xhi] \xhi^\top
\end{gather*}
\\
\textbf{(b)}
\\
(2) Optimizing wrt. $W,s^1,\ldots s^N$ and V is not a convex 
optimization problem while the approaches in task 1 and 2 are convex. 
The problem became harder to optimize. 
\\
\\
(1)
Infering the sources $\rhi$ from $\xhi$ is just the product $V^\top\xhi$ 
(requires dxh multiplications). This does not endanger the feasibility of this method.
\\
\\
(3)
Non-convexity (or non-concavity) generally endangers 
the feasibility of finding a solution. The computation generally becomes 
either harder or more approximative. In comparison, the 1-layer network 
is comparatively quick and exact solvable. 
\end{document}






