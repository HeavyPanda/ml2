\documentclass[a4paper]{article}
\usepackage{a4wide}
\usepackage{amsmath}
\usepackage{amssymb}
\usepackage{dsfont}
\pagestyle{plain}
\usepackage{tikz}
\usetikzlibrary{matrix}
\usepackage{eurosym}
\usepackage{amsthm}
\usepackage{mathrsfs}
\usepackage{dlfltxbcodetips}
\usepackage{cancel}
\usepackage{bbm}
\usepackage{graphicx}

%    \usepackage{tkz-graph}  
%    \usetikzlibrary{shapes.geometric}%   
% 	 \usetikzlibrary{bayesnet}
	
%\documentclass[12pt]{article}

\usepackage[utf8]{inputenc} % UTF8-Kodierung für Umlaute usw
% \usepackage[ngerman]{babel} % Sprache: Deutsch
\usepackage{graphicx}

\usepackage[margin=1in]{geometry}
\usepackage{array}
\usepackage{amsthm,amssymb}
%\usepackage[fleqn]{amsmath}

\usepackage{multicol}

\newcommand*{\carry}[1][1]{\overset{#1}}
\newcolumntype{B}[1]{r*{#1}{@{\,}r}}

\newcommand{\N}{\mathbb{N}}
\newcommand{\Z}{\mathbb{Z}}

\newcommand{\extitle}[1]{{\vspace{5mm}\Large\b{#1}}\vspace{2mm}\\}
\newcommand{\exsubtitle}[1]{{\large\b{#1}}\vspace{2mm}\\}

\renewcommand*{\i}[1]{{\textit{#1}}}
\renewcommand*{\b}[1]{{\textbf{#1}}}
\renewcommand*{\tt}[1]{{\texttt{#1}}}

\setlength{\parindent}{0cm}
%\setlength{\mathindent}{0cm}
 % import config
\DeclareMathOperator*{\argmax}{arg\,max}
\DeclareMathOperator*{\argmin}{arg\,min}  
\newcommand{\norm}[1]{\left\lVert#1\right\rVert}
\newcommand{\vivid}{\stackrel{\text{vivid}}{=}}
\newcommand{\icol}[1]{% inline column vector
  \left(\begin{smallmatrix}#1\end{smallmatrix}\right)%
}

\newcommand\indep{\protect\mathpalette{\protect\indeP}{\perp}}
\def\indeP#1#2{\mathrel{\rlap{$#1#2$}\mkern2mu{#1#2}}}

\newcommand{\mlq}{F(\q)}
\newcommand{\si}{s}
\newcommand{\sj}{{s'}}
\newcommand{\p}{p}
\newcommand{\q}{r}
\newcommand{\z}{w}
\newcommand{\e}{\textbf{e}}
%\newcommand{\x}{\mathcal{D}}
%\newcommand{\mlq}{-L(\q)}
%\newcommand{\si}{i}
%\newcommand{\sj}{j}
%\newcommand{\p}{p}
%\newcommand{\q}{q}
%\newcommand{\z}{Z}
%\newcommand{\x}{X}

\newcommand{\pxgz}{\p(\x|\z)}
\newcommand{\pzgx}{\p(\z|\x)}
\newcommand{\pxz}{\p(\x,\z)}
\newcommand{\px}{\p(\x)}

\newcommand{\qz}{\q(\z)}
\newcommand{\qi}{\q_\si}
\newcommand{\qj}{\q_\sj}
\newcommand{\qk}{\q_k}
\newcommand{\pz}{\p(\z)}
\newcommand{\pzi}{\p(\z_\si)}
\newcommand{\pzj}{\p(\z_\sj)}
\newcommand{\pzk}{\p(\z_k)}

\newcommand{\sigm}{\text{sigm}}

\newcommand{\zi}{\z_\si}
\newcommand{\zj}{\z_\sj}
\newcommand{\zk}{\z_k}
\newcommand{\dz}{\mathrm{d}\z}
\newcommand{\dzi}{\mathrm{d}\z_\si}
\newcommand{\dzj}{\mathrm{d}\z_\sj}
\newcommand{\dzk}{\mathrm{d}\z_k} 

\newcommand{\E}{\mathbb{E}}





%\newcommand{\propq}{ \underset{ \text{wrt.}\put(1,1){\scriptsize *}q }{\propto} }

\newcommand{\eqs}[1]{ \stackrel{#1}{=}  }
\newcommand{\eqq}{  \overset{\text{!}}{=} }
\newcommand{\propqz}{ \underset{ \overset{\text{wrt.}}{\qz} }{\propto} }
\newcommand{\propz}{ \underset{ \overset{\text{wrt.}}{\z} }{\propto} }
\newcommand{\propzj}{ \underset{ \overset{\text{wrt.}}{\zj} }{\propto} }
%\newcommand{\constw}{ \underset{ \text{ wrt. \w} }{\text{ const}} }
%\newcommand{\consty}{ \underset{ \text{ wrt. \y} }{\text{ const}} }
\newcommand{\const}[1]{{\underset{ \text{ wrt. #1} }{\text{ const}} }}

\newcommand{\tr}[1]{{#1^\top}}
\newcommand{\inv}[1]{{#1^{-1}}}
\newcommand{\trk}[1]{{(#1)^\top}}
\newcommand{\invk}[1]{{(#1)^{-1}}}
\newcommand{\tin}[1]{{(#1^\top)^{-1}}}

\newcommand{\Yt}{{Y^\top}}
\newcommand{\Xt}{{X^\top}}
\newcommand{\xt}{{\x^\top}}
\newcommand{\yt}{{\y^\top}}
\newcommand{\Wt}{{\W^\top}}
\newcommand{\w}{w}
\newcommand{\lx}{{\lambda_{x}}}
\newcommand{\ly}{{\lambda_{y}}}
\newcommand{\rhi}{{r^{(i)}}}
\newcommand{\rhj}{{r^{(j)}}}
\newcommand{\xhi}{{x^{(i)}}}
\newcommand{\shi}{{s^{(i)}}}
\newcommand{\xhj}{{x^{(j)}}}
\newcommand{\shj}{{s^{(j)}}}
\newcommand{\Wi}{{W_{i}}}
\newcommand{\Wj}{{W_{j}}}
\newcommand{\Wit}{W_{i}^\top}
\newcommand{\Wjt}{W_{j}^\top}
\newcommand{\Wij}{W_{ij}}
\newcommand{\Wlk}{W_{lk}}
\newcommand{\ax}{{\alpha_{x}}}
\newcommand{\ay}{{\alpha_{y}}}
\newcommand{\axt}{\alpha_{x}^\top}
\newcommand{\ayt}{\alpha_{y}^\top}
\newcommand{\Cxx}{C_{xx}}
\newcommand{\Cxy}{C_{xy}}
\newcommand{\Cyx}{C_{yx}} 
\newcommand{\Cyy}{C_{yy}}
\newcommand{\1}{\mathds{1}}
\newcommand{\lag}{\mathcal{L}}
\newcommand\bigforall{\mbox{\Large $\mathsurround0pt\forall$}} 

\begin{document}

% header configuration
\title{\b{Exercise Sheet 10}}
\author{Machine Learning 2, SS16}

\maketitle

% authors
Mario Tambos, 380599;\quad Viktor Jeney, 348969;\quad Sascha Huk, 321249;\quad Jan Tinapp, 0380549\\

\extitle{Exercise 1a}
For arbitrary n let $\vec{x}_1, \ldots, \vec{x}_n \in \mathcal{X}$ be arbitrary samples 
and $\vec{v}\in\mathbb{R}^n$ be an arbitrary vector.
\begin{gather*}
	\bigforall_{l=1}^L. \sum_{i,j=1}^n v_i v_j k_l(x_i, x_j) \geq 0 
	\Longrightarrow
	\sum_{l=1}^L\sum_{i,j=1}^n v_i v_j k_l(x_i, x_j) \geq 0
	\Longrightarrow
	\sum_{i,j=1}^n v_i v_j \sum_{l=1}^L k_l(x_i, x_j) \geq 0 \\
	\stackrel{\vec{\beta}\geqq 0}{\Longrightarrow}
	\sum_{i,j=1}^n v_i v_j \sum_{l=1}^L \beta_l k_l(x_i, x_j) \geq 0
	\stackrel{\text{Def. k}}{\Longrightarrow}
	\sum_{i,j=1}^n v_i v_j k(x_i, x_j) \geq 0	
\end{gather*}
%We could also prove this for all pairs of datapoints $x, x'$ out of $\mathcal{X}$, but then the unfold of the 
%definition of positive definiteness wouldn't be that lucid.

\extitle{Exercise 1b}
Let $x,x'\in \mathcal{X}$ be two arbitrary samples.
\begin{gather*}
	\sum_{l=1}^L \beta_l k_l(x,x') 
	= 
	\sum_{l=1}^L \beta_l \phi_l(x)^\top \phi_l(x') 
	=
	\sum_{l=1}^L (\sqrt{\beta_l} \phi_l(x)^\top) (\sqrt{\beta_l} \phi_l(x'))
	\\
	=
\underbrace{
\begin{bmatrix}
	\sqrt{\beta_1} \phi_1(x)^\top	& \dots	& \sqrt{\beta_L} \phi_L(x)^\top      
\end{bmatrix}
}_{\phi(x)^\top}
\underbrace{
\begin{bmatrix}
	\sqrt{\beta_1} \phi_1(x')	\\
	\dots	\\
	\sqrt{\beta_L} \phi_L(x')      
\end{bmatrix}
}_{\phi(x')}
\end{gather*}
$\phi(x)^\top$ and $\phi(x')$ are block partitioned matrices, 
i.e. the $\phi_l's$ are simply concatenated 
together in one very long vector. So, the result is: 
\begin{gather*}
	\phi(x) = 
	\begin{bmatrix}
		\sqrt{\beta_1} \phi_1(x)	\\
		\dots	\\
		\sqrt{\beta_L} \phi_L(x)      
	\end{bmatrix}
\end{gather*}
\newpage 
\extitle{Exercise 2a}
We also encode the classes via the canonical base vectors 
$\e_y$ for all $y\in \{1,\ldots,C\}$.
For arbitrary n let $\vec{x}_1, \ldots, \vec{x}_n \in \mathcal{X}$ be arbitrary samples 
and $\vec{v}\in\mathbb{R}^n$ be an arbitrary vector.
\begin{gather*}
	\sum_{i,j=1}^n v_i v_j k(x_i,x_j) \mathbbm{1}_{[y=y']}
	=
	\sum_{i,j=1}^n v_i v_j k(x_i,x_j) \e_y^\top \e_{y'} 
	\\
	%=
	%\sum_{i,j=1}^n v_i v_j \sum_k a_{ik}a_{jk} \e_y^\top \e_{y'}
	=
	\sum_{i,j} v_i v_j \phi(x_i)^\top \phi(x_j) \e_y^\top \e_{y'}
	=
	\sum_{i,j} v_i v_j \sum_k \phi(x_i)_k \phi(x_j)_k \e_y^\top \e_{y'}
	\\
	=
	\sum_k \sum_{i,j} v_i v_j \phi(x_i)_k \phi(x_j)_k \e_y^\top \e_{y'}
	=
	\sum_k \underbrace{(\sum_{i} v_i \phi(x_i)_k) (\sum_{j}v_j \phi(x_j)_k)}_{\text{sums are equal}} 
	\e_y^\top \e_{y'}
	\\
	=
	\sum_k \underbrace{(\sum_{i} v_i \phi(x_i)_k)^2}_{\geq 0}  \underbrace{\e_y^\top \e_{y'}}_{\geq 0}
	\geq 0
\end{gather*}

\extitle{Exercise 2b}
We also encode the classes via the canonical base vectors 
$\e_y$ for all $y\in \{1,\ldots,C\}$.
\begin{gather*}
	k(x,x')\mathbbm{1}_{[y=y']}
	=
	\phi(x)^\top \phi(x') \e^\top_y \e_{y'}
	%=
	%\e_{y}^\top (\phi(x)^\top \phi(x')) \e_{y'}
	=
	\sum_{i=1}^h\phi_{i}(x)\phi_{i}(x')\sum_{c=1}^C(\e_y)_c (\e_{y'})_c
	=
	\sum_{i=1}^h\sum_{c=1}^C\phi_{i}(x)\phi_{i}(x') (\e_y)_c (\e_{y'})_c
	\\
	=
	\sum_{c=1}^C \sum_{i=1}^h \phi_{i}(x)(\e_y)_c \phi_{i}(x') (\e_{y'})_c 	
	=
	\hspace{2mm}
	<
	\underbrace{
		\phi(x)\times \e_y 
	}_{\phi_{\text{struct}}(x,y)}
	,
	\hspace{2mm}
	\underbrace{
	\phi(x')\times \e_{y'}
	}_{\phi_{\text{struct}}(x',y')}
	>_F
\end{gather*}
$<.,.>_F$ is the Frobenius inner product. If one prefers
the standard scalar (and seeing $\phi_{struct}(x)$ as a vector) 
one could convert the matrices $\phi_{struct}$ into 
vectors by concatenating the row- or column vectors. Then using 
the standard scalar product on these vectors would yield the same result. 
  
\end{document}






