\documentclass[a4paper]{article}
\usepackage{a4wide}
\usepackage{amsmath}
\usepackage{amssymb}
\usepackage{dsfont}
\pagestyle{plain}
\usepackage{tikz}
\usetikzlibrary{matrix}
\usepackage{eurosym}
\usepackage{amsthm}
\usepackage{mathrsfs}
\usepackage{dlfltxbcodetips}
\usepackage{cancel}

%    \usepackage{tkz-graph}  
%    \usetikzlibrary{shapes.geometric}%   
% 	 \usetikzlibrary{bayesnet}
	
%\documentclass[12pt]{article}

\usepackage[utf8]{inputenc} % UTF8-Kodierung für Umlaute usw
% \usepackage[ngerman]{babel} % Sprache: Deutsch
\usepackage{graphicx}

\usepackage[margin=1in]{geometry}
\usepackage{array}
\usepackage{amsthm,amssymb}
%\usepackage[fleqn]{amsmath}

\usepackage{multicol}

\newcommand*{\carry}[1][1]{\overset{#1}}
\newcolumntype{B}[1]{r*{#1}{@{\,}r}}

\newcommand{\N}{\mathbb{N}}
\newcommand{\Z}{\mathbb{Z}}

\newcommand{\extitle}[1]{{\vspace{5mm}\Large\b{#1}}\vspace{2mm}\\}
\newcommand{\exsubtitle}[1]{{\large\b{#1}}\vspace{2mm}\\}

\renewcommand*{\i}[1]{{\textit{#1}}}
\renewcommand*{\b}[1]{{\textbf{#1}}}
\renewcommand*{\tt}[1]{{\texttt{#1}}}

\setlength{\parindent}{0cm}
%\setlength{\mathindent}{0cm}
 % import config
\DeclareMathOperator*{\argmax}{arg\,max}
\DeclareMathOperator*{\argmin}{arg\,min}  
\newcommand{\norm}[1]{\left\lVert#1\right\rVert}
\newcommand{\vivid}{\stackrel{\text{vivid}}{=}}
\newcommand{\icol}[1]{% inline column vector
  \left(\begin{smallmatrix}#1\end{smallmatrix}\right)%
}

\newcommand\indep{\protect\mathpalette{\protect\indeP}{\perp}}
\def\indeP#1#2{\mathrel{\rlap{$#1#2$}\mkern2mu{#1#2}}}

\newcommand{\mlq}{F(\q)}
\newcommand{\si}{s}
\newcommand{\sj}{{s'}}
\newcommand{\p}{p}
\newcommand{\q}{r}
\newcommand{\z}{w}
%\newcommand{\x}{\mathcal{D}}
%\newcommand{\mlq}{-L(\q)}
%\newcommand{\si}{i}
%\newcommand{\sj}{j}
%\newcommand{\p}{p}
%\newcommand{\q}{q}
%\newcommand{\z}{Z}
%\newcommand{\x}{X}

\newcommand{\pxgz}{\p(\x|\z)}
\newcommand{\pzgx}{\p(\z|\x)}
\newcommand{\pxz}{\p(\x,\z)}
\newcommand{\px}{\p(\x)}

\newcommand{\qz}{\q(\z)}
\newcommand{\qi}{\q_\si}
\newcommand{\qj}{\q_\sj}
\newcommand{\qk}{\q_k}
\newcommand{\pz}{\p(\z)}
\newcommand{\pzi}{\p(\z_\si)}
\newcommand{\pzj}{\p(\z_\sj)}
\newcommand{\pzk}{\p(\z_k)}



\newcommand{\zi}{\z_\si}
\newcommand{\zj}{\z_\sj}
\newcommand{\zk}{\z_k}
\newcommand{\dz}{\mathrm{d}\z}
\newcommand{\dzi}{\mathrm{d}\z_\si}
\newcommand{\dzj}{\mathrm{d}\z_\sj}
\newcommand{\dzk}{\mathrm{d}\z_k} 

\newcommand{\E}{\mathbb{E}}





%\newcommand{\propq}{ \underset{ \text{wrt.}\put(1,1){\scriptsize *}q }{\propto} }

\newcommand{\eqs}[1]{ \stackrel{#1}{=}  }
\newcommand{\eqq}{  \overset{\text{!}}{=} }
\newcommand{\propqz}{ \underset{ \overset{\text{wrt.}}{\qz} }{\propto} }
\newcommand{\propz}{ \underset{ \overset{\text{wrt.}}{\z} }{\propto} }
\newcommand{\propzj}{ \underset{ \overset{\text{wrt.}}{\zj} }{\propto} }
%\newcommand{\constw}{ \underset{ \text{ wrt. \w} }{\text{ const}} }
%\newcommand{\consty}{ \underset{ \text{ wrt. \y} }{\text{ const}} }
\newcommand{\const}[1]{{\underset{ \text{ wrt. #1} }{\text{ const}} }}

\newcommand{\tr}[1]{{#1^\top}}
\newcommand{\inv}[1]{{#1^{-1}}}
\newcommand{\trk}[1]{{(#1)^\top}}
\newcommand{\invk}[1]{{(#1)^{-1}}}
\newcommand{\tin}[1]{{(#1^\top)^{-1}}}

\newcommand{\Yt}{{Y^\top}}
\newcommand{\Xt}{{X^\top}}
\newcommand{\xt}{{\x^\top}}
\newcommand{\yt}{{\y^\top}}
\newcommand{\Wt}{{\W^\top}}
\newcommand{\w}{w}
\newcommand{\lx}{{\lambda_{x}}}
\newcommand{\ly}{{\lambda_{y}}}
\newcommand{\rhi}{{r^{(i)}}}
\newcommand{\rhj}{{r^{(j)}}}
\newcommand{\xhi}{{x^{(i)}}}
\newcommand{\shi}{{s^{(i)}}}
\newcommand{\xhj}{{x^{(j)}}}
\newcommand{\shj}{{s^{(j)}}}
\newcommand{\Wi}{{W_{i}}}
\newcommand{\Wj}{{W_{j}}}
\newcommand{\Wit}{W_{i}^\top}
\newcommand{\Wjt}{W_{j}^\top}
\newcommand{\Wij}{W_{ij}}
\newcommand{\Wlk}{W_{lk}}
\newcommand{\ax}{{\alpha_{x}}}
\newcommand{\ay}{{\alpha_{y}}}
\newcommand{\axt}{\alpha_{x}^\top}
\newcommand{\ayt}{\alpha_{y}^\top}
\newcommand{\Cxx}{C_{xx}}
\newcommand{\Cxy}{C_{xy}}
\newcommand{\Cyx}{C_{yx}} 
\newcommand{\Cyy}{C_{yy}}
\newcommand{\1}{\mathds{1}}
\newcommand{\lag}{\mathcal{L}}


\begin{document}

% header configuration
\title{\b{Exercise Sheet 5}}
\author{Machine Learning 2, SS16}

\maketitle

% authors
Mario Tambos, 380599;\quad Viktor Jeney, 348969;\quad Sascha Huk, 321249;\quad Jan Tinapp, 0380549\\


\extitle{Exercise 1 - Kernel Eigenvectors}
\textbf{(a)}\\
We use Lagrange multipliers:
\begin{gather*}
	\lag
	=
	v^\top Cv - \lambda(v^\top v - 1) 
	\\
	{\partial\lag\over\partial v} 
	=2(C - \lambda I)v \eqq \textbf{0} 
	\quad\Longrightarrow\quad Cv\eqq\lambda v
	\\
	{\partial\lag\over\partial v^2}
	=
	2(C-\lambda I)
\end{gather*}
In every matrix C it holds that $eig(C-a I) = (\lambda_i-a)_{i\in\{1,\ldots,d\}}$
(see eq. 286 in matrix cookbook) 
where in this case $a=\max(\{\lambda_i\}_{i\in\{1,\ldots,d\}})$. 
This means that the second order derivative has only negative eigenvalues 
except for the largest eigenvalue, which is now zero.   
Hence ${\partial\lag\over\partial v^2}$ is negative semi-definite (think 
of the bilinearform of combinations of eigenvectors, 
which correspond to the largest eigenvalue). 
We therefore use another simple criterion, namely using 
$v^\top v = 1$ and the necessary condition $Cv\eqq\lambda v$ in 
our objective function $v^\top C v$:
\begin{gather*}
	\max_v v^\top C v 
	=
	\max(\{v^\top C v\}_{v})
	=
	\max(\{v^\top \lambda_i v\}_{v,i})
	= 
	\max(\{\lambda_i\}_{i\in\{1,\ldots,d\}})
\end{gather*} 
Since the necessary condition wrt. $v^\top v=1$ describes an eigenvalue problem,
both cases are equivalent.  
\\
\\
\textbf{(b)}\\
\begin{gather*}
	\lambda v = Cv 
	\quad\Longleftrightarrow\quad
	v = {Cv\over\lambda} 
	\quad\stackrel{\text{def. C}}{\Longleftrightarrow}\quad
	v = \sum_{i=1}^N \phi(x_i)\underbrace{\phi(x_i)^\top v\over\lambda}_{\alpha_i} 
\end{gather*} 
Comparing the coefficients with the given equation
$v=\Phi^\top\alpha=\sum_{i=1}^N\phi(x_i)\alpha_i$, we get:
\begin{equation}
	{\phi(x_i)^\top v\over\lambda} = \alpha_i
	\quad\Longleftrightarrow\quad
	\phi(x_i)^\top v = \lambda\alpha_i
	\quad\Longleftrightarrow\quad
	\Phi v = \lambda\alpha
\end{equation}
We now conclude:
\begin{gather*}
	K\alpha = \Phi\Phi^\top\alpha = \Phi v \stackrel{(1)}{=} \lambda\alpha 
\end{gather*}

\extitle{Exercise 2}
Define \[ s:=(s_i)_{i=1,...,N}\] and the matrix of eigenvectors \[ U:=(u^T_i)_{i=1,...,N}. \]
Then \[ Uy=s. \]
Because all eigenvectors are orthogonal, $U$ is an orthogonal matrix. This means in particular, that \[||Uy||^2_2=||y||^2_2. \] 
All in all we get \[ \sum_{i=1}^{N}s_i=||s||^2_2=||Uy||^2_2=||y||^2_2 \]
\end{document}






